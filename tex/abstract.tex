Biological organisms are thought to possess traits that facilitate evolution.
The term evolvability was coined to describe this type of adaptation.
The question of evolvability has special practical relevance to computer science researchers engaged in longstanding efforts to harness evolution as an algorithm for automated design.
It is hoped that a more nuanced understanding of evolvability inspired by biological evolution will translate to more powerful digital evolution techniques.
To this end, the relationship between evolvability and environmental influence on the phenotype was investigated using digital experiments performed on a genetic regulatory model.
The phenotypic response of champion individuals evolved under regimes of direct plasticity, and indirect plasticity was assessed.
The model predicts that direct plasticity and indirect plasticity decrease and increase the frequency of silent mutations, respectively.